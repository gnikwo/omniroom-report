%!TEX program = xelatex
% not lualatex because of a pgf bug: https://sourceforge.net/p/pgf/bugs/384/
\documentclass[12pt, a4paper]{report}
\usepackage[T1]{fontenc}
\usepackage[french]{babel}
\usepackage{hyperref}
\usepackage{utbmcovers}
\usepackage{hyphenat}
\usepackage[scale=0.75]{geometry}
\usepackage{overpic}
\usepackage{tikz}
\usepackage{float}

\usetikzlibrary{arrows, positioning}

%----------------------------------------
% hyperref configuration
%----------------------------------------

\hypersetup{
    colorlinks=true,
    urlcolor=,
}

\graphicspath{{images/}}

\newcommand\tab[1][1cm]{\hspace*{#1}}
\newcommand\TODO[1]{\textcolor{red}{TODO\@: #1}}

%----------------------------------------
% utbmcovers configuration
%----------------------------------------

\setutbmfrontillustration{utbm_default_illustration}
\setutbmtitle{Développement d'un système de video-surveillance à faible latence}
\setutbmsubtitle{Rapport de travail complémentaire \hyp{} A2019}
\setutbmstudent{Nicolas BALLET}
\setutbmstudentdepartment{Département Génie Informatique}
\setutbmstudentpathway{Filière libre}
\setutbmcompany{ }
\setutbmcompanyaddress{ }
\setutbmcompanywebsite{ }
\setutbmcompanytutor{ }
\setutbmschooltutor{Frank Gechter}
\setutbmkeywords{Video surveillance \hyp{} Raspberry pi \hyp{} Gstreamer \hyp{} Github \hyp{} Reconnaissance faciale}
\setutbmabstract{J'ai eu l'occasion en paralléle de mon stage de fin d'étude de concevoir et de commencer à produire une solution de video-surveillance à faible coût, à faible latence et facile à déployer.}

%----------------------------------------
% document
%----------------------------------------

\begin{document}
\makeutbmfrontcover{}
\tableofcontents
\chapter{Introduction}
Besoin:\newline
    Au meme endroit je veux pouvoir voir un ensemble de caméra regroupées sous forme de groupes.\newline
    Aucune solution simple, légère, à faible latence et opensource sur le marché.\newline
    Solution simple avec Gstreamer App -> App. Pas pratique, obligé d'avoir l'appli.\newline
    Proposition: Utiliser un navigateur comme client.\newline\newline
Conception:\newline
    Quels protocoles ? <Liste des protocoles de streaming> + choix du WebRTC\newline
    Quels encodage ? Compatibilité navigateur: VP8/VP9 - H264\newline
    Choix du H264 car encodage matériel plus répandu, mais possibilité de modification.\newline
    Quelle technologie coté camera ? Besoin de rapidité, C / C++, Gstreamer, choix du C++ par compétences.\newline\newline
Développement:\newline
    Premier POC\newline
    Refonte C++\newline
    Définition d'un protocole métier\newline

\makeutbmbackcover{}
\end{document}
